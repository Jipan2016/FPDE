\chapter{有限差分法}
  
\begin{introduction}   
\item 特殊函数
\item 分数阶微积分的定义
\item 分数阶微积分的基本性质
\end{introduction}

\section{差分法基本概念}
\subsection{有限差分法的数值差分}

定义当$x=x_i$,$u(x_i) = u_i$。即下标$i$表示$x=x_i$时$u(x)$的值。类似的$u_{i,j,k}^{n}$表示$x=x_i,y=y_j,z=z_k,t=t_n$时$u(x,y,z)$的值。有时候上标$n$也表示迭代次数,具体问题看是否出现时间项。若出现时间项,一般是表示$t=t_n$,也有可能有双上标,即既有时间项也有迭代次数。若没有出现时间项,则必然表示迭代次数。

对于一维问题,求解区域为$x \in [a,b] $,这里讲区间等分为$N$等分,则$x$方向步长为$\Delta x = (b-a)/N$,则
$$x_i = a+i\Delta x,i=0,1,2,\cdots, N$$
常见的一维差商问题有一下几种类型:

(1)一阶导数向前差商:误差为$ \mathcal{O}(\Delta x)$,其中$0 \le i \le N-1$.
\begin{equation}
	u'(x_i) = \frac{1}{\Delta x}\left[u(x_{i+1})-u(x_i)\right] \Leftrightarrow u'_i  = \frac{1}{\Delta x}\left[u_{i+1}-u_i\right]
\end{equation} 


\section{差分法二维扩散方程中的应用}

\subsection{一维扩散方程}

有限域的一维扩散方程的表达式为
$$ \frac{\partial u}{\partial t} = \alpha \frac{\partial u^2}{\partial x^2}$$

\subsection{1D Crank-Nicolson method} 
The Crank–Nicolson method ~\cite{Wiki-Crank-Nicolson-Method} is often applied to diffusion problems. As an example, for linear diffusion,
$$ {\partial u \over \partial t}=\alpha {\frac  {\partial ^{2}u}{\partial x^{2}}} $$
applying a finite difference spatial discretization for the right hand side, the Crank–Nicolson discretization is then:
$${\frac{u_{i}^{n+1}-u_{i}^{n}}{\Delta t}}={\frac{\alpha}{2(\Delta x)^{2}}}((u_{i+1}^{n+1}-2u_{i}^{n+1}+u_{i-1}^{n+1})+(u_{i+1}^{n}-2u_{i}^{n}+u_{i-1}^{n}))$$
or, letting ${\displaystyle r={\frac {\alpha \Delta t}{2(\Delta x)^{2}}}}$:
$$-ru_{{i+1}}^{{n+1}}+(1+2r)u_{{i}}^{{n+1}}-ru_{{i-1}}^{{n+1}}=ru_{{i+1}}^{{n}}+(1-2r)u_{{i}}^{{n}}+ru_{{i-1}}^{{n}}$$
given that the terms on the right-hand side of the equation are known, this is a tridiagonal problem, so that ${\displaystyle u_{i}^{n+1}\,}$, may be efficiently solved by using the tridiagonal matrix algorithm in favor of a much more costly matrix inversion.

A quasilinear equation, such as (this is a minimalistic example and not general)
$${\frac  {\partial u}{\partial t}}=\alpha(u){\frac  {\partial ^{2}u}{\partial x^{2}}}$$
would lead to a nonlinear system of algebraic equations which could not be easily solved as above; however, it is possible in some cases to linearize the problem by using the old value for $\alpha$, that is $\alpha_{{i}}^{{n}}(u)$, instead of $ \alpha_{{i}}^{{n+1}}(u)$. Other times, it may be possible to estimate $\alpha_{{i}}^{{n+1}}(u)$, using an explicit method and maintain stability.

\subsection{Leap-Frog method}



\section{差分法二维扩散方程中的应用}

\subsection{二维扩散方程}

有限域的二维扩散方程的是一个在空间上为椭圆型微分方程,在空间上为双曲型微分方程,其典型的表达式为
$$ \frac{\partial u}{\partial t} = \alpha \nabla ^2 u = \alpha \Delta u = \alpha (\frac{\partial u^2}{\partial x^2}+\frac{\partial ^2 u}{\partial y^2}),\qquad \alpha>0 $$

Domain: $(x,y) \in \Omega \subset \mathbb {R}^2 $ and $t>0$,$\Gamma_D$ is the boundary of $\Omega$.

Initial condtion:
\begin{itemize}
	\itemsep=4pt
	\parskip=2pt
	\item (1) Dirichlet initial condition: 
	$$u(t=0,x,y)|_{(x,y) \in \Omega}=\varphi_0(x,y)$$
	\item (2) Neumann initial condition:
	 $$ \vm{D} \cdot \nabla u(t=0,x,y)|_{(x,y) \in \Omega} = \frac{\partial u}{\partial \vm{n}} = q_0(x,y)$$
	\item (3) Robin initial condition:
	$$ a u(t=0,x,y)_{(x,y) \in \Gamma_D} +b \frac{\partial u(t=0,x,y)}{\partial \vm{n}}_{(x,y) \in \Gamma_D} = g(x,y)$$
\end{itemize}
	
Boundary condition:
\begin{itemize}
	\itemsep=4pt
	\parskip=2pt
	\item (1) Dirichlet boundary condition: 
	$$u(t,x,y)|_{(x,y) \in \Gamma_D}=\varphi_0(x,y)$$
	\item (2) Neumann boundary condition:
	$$ \vm{D} \cdot \nabla u(u,x,y)|_{(x,y) \in \Gamma_D} = \frac{\partial u}{\partial \vm{n}} = q_0(x,y)$$
	\item (3) Robin boundary condition:
	$$ a u(t,x,y)_{(u,x,y) \in \Gamma_D} +b \frac{\partial u}{\partial \vm{n}}_{(x,y) in \Gamma_D} = g(x,y)$$
	\item (4) Mixed boundary condition:

	\item (5) Cauchy boundary condition:
	
\end{itemize}	
	
	
	
	
	
	 
\subsection{Crank-Nicolson method}



\subsection{Leap-Frog method}


