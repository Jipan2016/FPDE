\chapter{数理方程的基础理论}

\begin{introduction}   
	\item 矢量分析与场论
	\item 函数空间
	\item 线性微分算符
	\item 线性微分算符的本征值问题
	\item 广义函数
\end{introduction}

\section{矢量分析与场论}

\subsection{矢量分析}

\subsection{场论}

\subsection{哈密顿算子}

\subsection{正交坐标系}


\section{函数空间}


\subsection{度量空间与赋范线性空间}

\subsubsection{度量空间}
	
	矢量是线性代数中的基本概念,作为构成线性空间的元素。矢量具有类似于正常三维矢量的代数特性,即矢量具有加法和数乘两种运算,并且遵从相对应的运算法则。但是它并具有长度和方向的集合特征。因此这里作为数学抽象的线性空间也能附加上于三维矢量分析类似的直观的几何结果。为此首先引入距离的概念。	

	\begin{definition}{度量空间与度量函数}{def2-3-2-1}
		元素(称为\hint{点}) $x$,$y$,$z$,$\cdots$的非空集合$\mathscr{X}$称为\hint{度量空间}(或\hint{距离空间}),如果对于每一对元素$x,y,z$,存在与它们相联系的一个实数$d(x,y)$,满足下列三个条件
		\begin{enumerate}
			\item (对称性)元素A到B之间的距离等于元素B到A之间的距离。
			$$d(x,y) = d(y,x);$$
			\item (非负性)元素之间的距离大于等于0,若距离等于0则为相同元素。
			$$ d(x,y) \ge 0, \quad d(X,Y)=0 \Leftrightarrow x=y;$$
			\item  (三角形不等式)元素之间的距离满足三角形不等式。
			$$d(x,y) \le d(x,z)+d(y,z).$$
		\end{enumerate}
		或可以理解为一个度量空间(metric space)由一个有序对(ordered pair) $(\mathscr{X},d)$ 表示,其中$\mathscr{X}$是一种集合
		,$d$是定义在$(\mathscr{X},d)$上的一种度量,是如下的一种函数映射:
		$$ \mathscr{X} \times \mathscr{X} \rightarrow \mathbb{R}.$$
		其满足上述三条性质,则称$(\mathscr{X},d)$度量空间或距离空间。$\mathscr{X}$中的元素称为点,点$P_0$的$\delta$邻域和点$P_0$的$\delta$去心邻域为
		\begin{equation}
		U(P_0,\delta) = \left\{P|d(P,P_0)<\delta \right\},\quad \stackrel{\circ}{U}(P_0,\delta) = \left\{P|0<d(P,P_0)<\delta \right\},
		\end{equation}	
		点$P_0$为邻域的中心,$\delta$称为邻域的半径。
	\end{definition}

	\begin{note}
		\begin{enumerate}
			\item 度量空间的元素(点)可以不是矢量。距离不仅仅是点与点之间的距离,除此之外还有向量之间的距离,曲线之间的距离,函数之间的距。这儿谈到距离的定义是一种泛指的概念。
			\item 无需定义元素的加法和数乘运算,因此度量空间可以不是线性空间,换言之,线性空间是度量空间的子集。
		\end{enumerate}
	\end{note}

	\begin{example}
		常见的度量空间如下:
		\begin{enumerate}
		\item 
			所有实数的集合(实轴)构成度量空间(记为$\mathscr{R}_1$),如果两个实数(实轴上的两点)$x$和$y$之间的距离定义为$|x-y|$.
		\item 
			所有复数$z = x+iy$的集合构成度量空间(记为$\mathscr{C}_1$),若两个复数$z_1$和$z_2$之间的距离定义为
			$$ d(z_1,z_2) = |z_1-z_2| = \sqrt{(x_1-x_2)^2+(y_1-y_2)^2}.$$
			\item 所有$n(n\in \mathbb{N^+})$个有序实数构成度量空间为$n$维欧式空间,记为$\mathscr{A}^{(r)}_{n}$。令
			$$x = (\xi_1,\xi_2,\cdots,\xi_n), \quad y = (\eta_1,\eta_2,\cdots,\eta_n)$$
			是空间中的两个点,引入不同的距离定义,就得到不同的度量空间,例如,可以定义距离函数为
			\begin{equation}
				d_p(x,y) = (|\xi_1-\eta_1|^p+|\xi_2-\eta_2|^p+\cdots+|\xi_n-\eta_n|^p)^{\frac{1}{p}} = [\sum_{i=1}^{n}(|xi_i-\eta_i|^p]^{\frac{1}{p}},\quad 1\le p\le \infty
			\end{equation}
			包括了取 $p = 1,2,\infty$的情形分别为直线距离,折线距离和最大距离:
			\begin{equation}
			\left\{  
			\begin{aligned}
			{}&d_1(x,y) = |\xi_1-\eta_1|+|\xi_2-\eta_2|+\cdots+|\xi_n-\eta_n| = \sum_{i=1}^{n}|\xi_i-\eta_i|\\
			{}&d_2(x,y) = (|\xi_1-\eta_1|^2+|\xi_2-\eta_2|^2+\cdots+|\xi_n-\eta_n|^2)^{\frac{1}{2}} = [\sum_{i=1}^{n}|\xi_i-\eta_i|^2]^{\frac{1}{2}}\\
			{}&d_\infty(x,y) = \max_i|\xi_i-\eta_i|\\
			\end{aligned}
			\right. 
			\end{equation}
		\item 
			所有$n(n\in \mathbb{N^+})$个有序复数构成度量空间为$n$维复空间,记为$\mathscr{A}^{(c)}_{n}$。
		\item 设$(\mathscr{X},d)$为度量空间,对$\mathscr{X}$中的任意两元素$x,y \in \mathscr{X}$,令
				\begin{equation}
					d(x,y)  = \left\{ 
					\begin{aligned}
					{}&1,\quad {\rm if ~}  x \neq y\\
					{}&0,\quad {\rm if ~}  x = y\\
					\end{aligned}
					\right. 
				\end{equation}
			这样的度量空间$(\mathscr{X},d)$称为离散的度量空间。
		\item 设$\mathcal{S}$为表示为实数列(复数列)的全体,对$\mathcal{S}$中任意两元素$x$,$y$
			$$x = (\xi_1,\xi_2,\cdots,\xi_n,\cdots), \quad y = (\eta_1,\eta_2,\cdots,\eta_n,\cdots)$$
			\begin{equation}
			d(x,y)  = \sum_{i=1}^{\infty} \frac{1}{2^i}\frac{|\xi_i-\eta_i|}{1+|\xi_i-\eta_i|}
			\end{equation}
			这样的度量空间$(\mathcal{S},d)$称为序列空间。
		\item
			设$\mathcal{A}$是一个给定的集合,令$\mathcal{B}(\mathcal{A})$表示$\mathcal{A}$上有界实值(或复值)函数全体,对$\mathcal{B}(\mathcal{A})$中任意两点$x$,$y$,定义
			\begin{equation}
				d(x, y)=\sup _{t \in \mathcal{A}}|x(t)-y(t)|
			\end{equation}
			这样的度量空间$(\mathcal{B}(\mathcal{A}),d)$称为有界函数空间。
		\item
			设$\mathcal{M}$为$\mathcal{X}$上实值(或复值)的勒贝格可测函数全体,$m$为勒贝格测度,若$m(X)<\infty$,对任意两个可测函数$f(t)$及$g(t)$,由于$ \displaystyle \frac{|f(t)-g(t)|}{1+|f(t)-g(t)|}<1 $,所以这就是$\mathcal{X}$上的可积函数。定义
			\begin{equation}
				d(f, g)= \int_{\mathcal{X}}\frac{|f(t)-g(t)|}{1+|f(t)-g(t)|}dt
			\end{equation}
		\item
			令$C[a,b]$表示闭区间$[a,b]$上实值(或复值)连续函数全体,对$C[a,b]$中任意两点$x$,$y$,定义
			\begin{equation}
				d(x, y)=  \max_{a\le t \le b} |x(t)-y(t)|
			\end{equation}
			这样的度量空间$(C[a,b],d)$称为$C[a,b]$空间。
		\item
			令$l^p$为集合$\displaystyle l^p =\{ x = (x_1,x_2,\cdots,x_i,\cdots ) |\sum_{i=1}^{\infty}x^p_k < \infty \}$,其中$x$,$y$为$l^p$中的两个元素,定义
			\begin{equation}
			d(x,y) = [\sum_{i=1}^{n}(|x_i-y_i|^p]^{\frac{1}{p}}
			\end{equation}
			这样的度量空间$l^p$空间。
		\end{enumerate}
	\end{example}

\subsubsection{度量空间中的极限、稠密集、可分空间}

	\begin{definition}{收敛点列}{def2-3-2-2}
		设$\{x_k\}$ 是$(\mathcal{X},d)$中的点列,如果任给$\epsilon>0$,存在指标$N$,使得
		$$d(x,x_k) \le \varepsilon, \quad  \forall k\le N \Leftrightarrow \lim _{k \rightarrow \infty} d\left(x_{k}, x\right)=0$$
		则称点列$\{x_k\}$是$(\mathcal{X},d)$中的\hint{收敛点列},点列$\{x_k\}$收敛到$x$ $(x_k \rightarrow x )$,或称$\{x_k\}$的极限为$x$,记为$\displaystyle \lim _{k \rightarrow \infty} x_{k}=x$。
	\end{definition}

	\begin{property}\label{property:converge_series}
		收敛点列的性质
		\begin{enumerate}
			\item 在度量空间中,任何一个点列最多只有一个极限,即收敛点列的极限是唯一的。
			\item $\mathcal{M}$是闭集的充要条件是$\mathcal{M}$中任何收敛点列的极限都在M中。
		\end{enumerate}
	\end{property}

	\begin{example}
		常见的度量空间的极限如下:
		\begin{enumerate}
			\item $n$维欧式空间中,$\{x_m\}$按欧式距离收敛于的充要条件是$x_m$依坐标收敛$x$,其中$x_m = (\xi_1^{(m)},\xi_2^{(m)},\cdots,\xi_n^{(m)})$,其中$m=1,2,\cdots$为$\mathbb{R}$的点列,$x = (\xi_1^{(m)},\xi_2^{(m)},\cdots,\xi_n^{(m)})$为点列$\{x_m\}$极限,则
			$$\lim _{m \rightarrow \infty} d\left(x_{m}, x\right)=0 \Leftrightarrow \lim _{m \rightarrow \infty} x_{m}=x \Leftrightarrow \lim _{m \rightarrow \infty} \xi ^{(m)}_i= \xi_i,i=1,2,\cdots m.$$
			\item 在序列空间序列$\mathcal{S}$中,$x_m = (\xi_1^{(m)},\xi_2^{(m)},\cdots,\xi_n^{(m)})$,其中$ m=1,2,\cdots$,序列$\{x_m\}$极限为$x = (\xi_1^{(m)},\xi_2^{(m)},\cdots,\xi_n^{(m)}) \in \mathcal{S} $,则
			$$\lim _{m \rightarrow \infty} d\left(x_{m}, x\right)=0 \Leftrightarrow \lim _{m \rightarrow \infty} x_{m}=x \Leftrightarrow \lim _{m \rightarrow \infty} \xi ^{(m)}_i= \xi_i,i=1,2,\cdots m.$$
			\item 设$\{ x_n\}$及$x$分别为$C[a,b]$中的点列及点,$d(x_m, x)=  \max_{a\le t \le b} |x_m(t)-x(t)|$,则
			$$\lim _{m \rightarrow \infty} d\left(x_{m}, x\right)=0 \Leftrightarrow \{x_n\} \mbox{在} [a,b] \mbox{上一致收敛于} x.$$
			\item 设$\{f_n\}$及$f$分别为可测函数空间中的点列及点,则
			$$\lim _{m \rightarrow \infty} d\left(f_{m}, x\right)=0 \Leftrightarrow \lim _{m \rightarrow \infty} f_n(t)= f(t).$$
		\end{enumerate}
	\end{example}
	
	\begin{definition}{有界集}{def2-3-2-3}
		设$\mathcal{M}$是度量空间$(\mathcal{X},d)$中点集,定义$\delta(\mathcal{M}) = \sup _{x,y \in \mathcal{X}} d(x,y)$为点集$\mathcal{M}$的直径,若$\delta(\mathcal{M}) <\infty$,则称$\mathcal{M}$为$(\mathcal{X},d)$中的有界集。度量空间中的收敛点列是\hint{有界点集}。	
	\end{definition}

	\begin{definition}{稠密集}{def2-3-2-4}
		设$\mathcal{X}$是度量空间,$\mathcal{E}$和$\mathcal{M}$是$\mathcal{X}$中的两个子集,令$\overline{\mathcal{M}}$表示$\mathcal{M}$的闭包,如果,$\mathcal{E} \subset \overline{\mathcal{M}}$,那么称集$\mathcal{M}$在集$\mathcal{E}$中\hint{稠密}。即如果$\mathcal{E}$中任何一点$x$的任何邻域都含有集$\mathcal{M}$中的点,就称$\mathcal{M}$在$\mathcal{E}$中稠密。
	\end{definition}


	\begin{note}
		\begin{enumerate}
			\item (\hint{稠密集的极限})对任一$x \in \mathcal{E}$,有$\mathcal{M}$中的点列$\{x_k\}$,使得$\displaystyle \lim _{k \rightarrow \infty} x_{k}=x$.
			\item (\hint{稠密子集})当$\mathcal{E}=\mathcal{X}$时,称集$\mathcal{M}$为$\mathcal{X}$的一个稠密子集。
			\item (\hint{可分空间}) 如果$\mathcal{X}$有一个可数的稠密子集时,称$\mathcal{X}$为可分空间。
		\end{enumerate}
	\end{note}

	\begin{example}
		常见的稠密集如下
		\begin{enumerate}
			\item 多项式全体所成的线性空间$\mathcal{P}$是度量空间$C[a,b]$ 的子集,则$\mathcal{P}$在$C[a,b]$ 中是稠密的。其中,以有理数为系数的多项式全体是一个可数集,所以$C[a,b]$是可分空间。
			\item $n$ 维欧式空间$\mathbb{R}^n$是可分空间,因为坐标为有理数的全体是一个可数集,是$\mathbb{R}^n$中的稠密子集。
			\item $l^p$为可分空间。$l^\infty$为不可分空间。
			令$l^p$为集合
			$$\displaystyle l^p =\{ x = (x_1,x_2,\cdots,x_i,\cdots ) |\sum_{i=1}^{\infty}x^p_k < \infty \}$$
			其中$x$,$y$为$l^p$中的两个元素,定义
			\begin{equation}
			d(x,y) = [\sum_{i=1}^{n}(|x_i-y_i|^p]^{\frac{1}{p}}
			\end{equation}
			这样的度量空间$l^p$空间。
		\end{enumerate}
	\end{example}

\subsubsection{连续映射}

	\begin{definition}{度量空间中的连续性}{def2-3-2-5}
		设$(\mathcal{X},d)$和$(\mathcal{Y},\tilde{d})$是两个度量空间,$T$是$\mathcal{X}$到$\mathcal{Y}$中的映射,$x_0 \in \mathcal{X}$ ,如果对于任意给定$\varepsilon>0$,存在$\delta>0$,使对$\mathcal{X}$中一切满足$d(x,x_0)<\delta$的$x$,有$\tilde{d}(Tx,Tx_0)<\varepsilon$成立,则称$T$在$x_0$处连续。	
	\end{definition}

	使用$\varepsilon-\delta$语言表述度量空间中的连续性为:
	$$T \mbox{ 在 }x_0\mbox{ 处连续 }\Leftrightarrow \forall U (Tx_0,\varepsilon)\mbox{ 必有 } V(x_0,\delta),\mbox{ 使得 } TV \in U.$$
	
	连续性定义也可以通过度量空间的极限进行定义。设$T$是度量空间$(\mathcal{X},d)$和$(\mathcal{Y},\tilde{d})$的映射,那么$T$在$x_0 \in \mathcal{X}$连续的充要条件为当$\displaystyle \lim _{k \rightarrow \infty} x_{k}=x_0$时,必有$\displaystyle \lim _{k \rightarrow \infty} Tx_{k}=Tx_0$。显示在度量空间在$x_0 \in \mathcal{X}$有极限是度量空间在$x_0 \in \mathcal{X}$连续的必要不充分条件。

	\begin{definition}{连续映射}{def2-3-2-6}
		如果映射$T$在$\mathcal{X}$的每一点都连续,则称$T$是$\mathcal{X}$上的连续映射。称集合$\{x | x \in X, T x \in M \subset Y\}$为集合$\mathcal{M}$在映射$T$下的原像。
	\end{definition}

	\begin{theorem}{连续映射的充要条件}{theo2-3-2-1}
	度量空间$\mathcal{X}$到$\mathcal{Y}$的映射T是$\mathcal{X}$上的连续映射的充要条件为$\mathcal{Y}$中任意开集$\mathcal{M}$的原像$T^{-1}M$ 是$\mathcal{X}$中的开集。
	\end{theorem}

\subsubsection{柯西点列和完备度量空间}


欧几里得空间,希尔伯特空间,巴拿赫空间或者是拓扑空间都属于函数空间。函数空间 = 元素 + 规则 ,即一个函数空间由 元素 与 元素所满足的规则 定义,而要明白这些函数空间的定义首先得从距离,范数,内积,完备性等基本概念说起。

二.范数
$$
范数 是比 距离 限制条件更多的一个概念。为了形象地解释范数的概念,这儿在二维平面进行说明。
在定义了 距离 这个概念之后,我们便可以描述二维平面上两个点之间的 距离 ,此时这个空间称作 度量空间 。但目前的条件没有办法描述一个点的“长度” ,因为缺少了 零点 。而范数定义之后此空间便多了一个零点,可以联想我们熟悉的平面直角坐标系,二维平面中范数可以看做是平面中的点到零点的距离。拥有范数的空间称作赋范空间,用符号∣∣X∣∣ ||X||∣∣X∣∣表示元素X XX的范数。因为范数的概念是在距离的概念上加了新的限制,则赋范空间一定是度量空间。我们可以用范数定义距离:
d(X,Y)=∣∣X−Y∣∣ d(X,Y)=||X-Y||
d(X,Y)=∣∣X−Y∣∣
$$
总结:元素X XX的范数∣∣X∣∣ ||X||∣∣X∣∣可简单看做X XX到零点的近距离。

三.线性
线性这个概念可以说是很熟悉了,即为加法与乘法的结合。若一个空间为线性空间,只要我们知道了此空间的所有基,便可以用加法与数乘表示这一空间所有的元素,如二维平面中能用X轴的单位向量与Y轴的单位向量表示此平面的任意向量。

$$
四.内积
内积又称点积或者数量积,在高中学习向量的点乘运算时便接触到这一概念。在有了前面的定义之后的空间总觉得与我们最熟悉的空间还差点什么,没错,就是角度。在引入内积之后的空间便有了角度的概念。X XX与Y YY的内积用符号(X,Y) (X,Y)(X,Y)表示,内积的结果同样是为实数。内积是在范数的概念上加了更多限制条件,即内积空间一定为赋范空间,同样的,可以用内积定义范数如下:
∣∣X∣∣2=(X,X) ||X||^2 = (X,X)
∣∣X∣∣ 
2
=(X,X)
$$

目前为止便完成了本文的大部分内容,有限维内积空间便是我们最熟悉的欧几里得空间。


$$
五.完备性
完备性这个概念的历史渊源比较深厚,作为非数学专业的工科生我也不太明白完备性的具体含义,简单来说对集合中的元素取极限不超出此空间便称其具有完备性。可以反向地通过不完备来理解完备性,对于整数集而言,对5取极限,便会超出整数集,即整数集不完备。
2018-10-22更正: 最近学了一点泛函,对完备性有了新的理解。完备性是在极限的基础上衍生的概念。例如在有理数集上的一个序列{1,1.4,1.41,1.414,1.4142…},可知此序列极限为2–√ \sqrt{2} 
2	
,而2–√ \sqrt{2} 
2	
为无理数,不属于有理数集,即有理数集不具备完备性。
$$

有了以上的概念理解众多迷糊人的空间便容易得多了

线性完备内积空间称作希尔伯特空间
线性完备赋范空间称作巴拿赫空间
有限维线性内积空间称作欧几里得空间
需要更加深入地理解希尔伯特空间大概避不开泛函分析,但作为工科学生,大概了解其概念够用就好。

线性空间

欧式空间

希尔伯特空间

\url{https://blog.csdn.net/qq\_34099953/article/details/84190508}

\url{http://open.163.com/movie/2013/3/T/0/M8PTB0GHI_M8PTBUHT0.html}