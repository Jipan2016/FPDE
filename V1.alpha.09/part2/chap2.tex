\chapter{分析力学基础}

\begin{introduction}   
	\item 虚位移原理
	\item 自由度和广义坐标
	\item 以广义坐标表示的质点系平衡条件 
	\item 动力学普遍方程
	\item 第二类拉格朗日方程
	\item 拉格朗日方程的初积分 
	\item 第一类拉格朗日方程	
\end{introduction}

18世纪提出了处理多个约束的刚体系统动力学问题。利用矢量力学分析出现以下问题:

(1)对于复杂约束系统约束力的性质和分布是未知的;

(2)表述形式复杂。如球坐标系下的运动方程;

(3)质点系问题为大量方程的微分方程组。            

1788年拉格朗日发表了《 分析力学》一书,提出了解决动力学问题的新观点和新方法:\hint{采用功和能量来描述物体的运动和相互作用力之间的关系}。

与矢量力学相比,分析力学的特点:    

(1)把约束看成对系统位置(速度)的限定,而不是看成一种力。

(2)使用广义坐标、功、能等标量研究系统运动,大量使用数学分析方法,得到标量方程。

(3)追求一般理论和一般模型,对于具体问题,只要代入和展开的工作,处理问题规范化。

(4)不仅研究获得运动微分方程的方法,也研究其求解的一般方法。

\section{虚位移原理}

\subsection{约束及其分类}

限制质点或质点系运动的条件称为\hint{约束}。限制条件的数学方程称为\hint{约束方程}。

(1)几何约束和运动约束:限制质点或质点系在空间的几何位置的条件称为几何约束。限制质点系运动情况的运动学条件称运动约束。

(2)定常约束和非定常约束:不随时间变化的约束称定常约束。约束条件随时间变化的称非定常约束。

(3)其它分类:约束方程中包含坐标对时间的导数,且不可能积分成有限形式的约束称非完整约束。约束方程中不包含坐标对时间的导数,或者约束方程中的积分项可以积分为有限形式的约束为完整约束。约束方程是等式的,称双侧约束(固执约束)。约束方程为不等式的,称单侧约束(非固执单侧约束)。

通常的约束为定常的双侧、完整、几何约束。即$n$个质点,$s$个约束。
$$f_i(x_1,y_1,z_1,\cdots,x_n,y_n,z_n) =0, \qquad i=1,2,\cdots s $$

\subsection{虚位移·虚功}

(1)虚位移:在某瞬时,质点系在约束允许的条件下,可能实现的任何无限小的位移称为虚位移。只与约束条件有关。通常表示为$\delta \vec{r}$,$\delta x$,$\delta \varphi$。

(2) 虚功:力在虚位移上作的功称虚功。
$$\delta W = \vec{F}\cdot \delta \vec{r} \qquad \delta W = M \delta \varphi$$
\subsection{理想约束}

如果在质点系的任何虚位移中,所有约束力所作虚功的和等于零,称这种约束为理想约束。即

$$\delta W_N = \sum \delta W_{Ni} = \Sigma \vec{F}_{Ni}\cdot \delta \vec{r}_i$$

光滑固定面约束、光滑铰链、无重刚杆,不可伸长的柔索、固定端等约束为理想约束。

\subsection{虚位移原理}

设质点系处于平衡,有
$${{\vec{F}}_{i}}+{{\vec{F}}_{\text{N}i}}=0 \qquad  {{\vec{F}}_{i}} \cdot \delta  {{\vec{r}}_{i}} +{{\vec{F}}_{\text{N}i}} \cdot \delta {{\vec{r}}_{i}}={{\vec{F}}_{i}} \cdot \delta  {{\vec{r}}_{i}} = 0 $$
或记为 $ \sum{\text{ }\!\!\delta\!\!\text{ }{{W}_{Fi}}}=0 $。此方程称虚功方程,其表达的原理称虚位移原理或虚功原理。

对于具有理想约束的质点系,其平衡的充分必要条件是:作用于质点系的所有主动力在任何虚位移中所作的虚功的和等于零。解析式为
$$\sum{\left( {{F}_{xi}}\text{ }\!\!\delta\!\!\text{ }{{x}_{i}}+{{F}_{yi}}\text{ }\!\!\delta\!\!\text{ }{{y}_{i}}+{{F}_{zi}}\text{ }\!\!\delta\!\!\text{ }{{z}_{i}} \right)=0}$$

\section{自由度和广义坐标}

\subsection{自由度}

\begin{definition}{Degree of Freedom}{int}
	在完整约束的条件下,确定质点系位置的\hint{独立参数}的数目,称为质点系的\hint{自由度数},简称\hint{自由度}。
\end{definition}

\begin{example}
	确定一个质点在空间的位置需3个独立的参量,自由质点为3个自由度。
\end{example}
\begin{example}	
	确定一个质点在平面的位置需2个独立的参量,平面自由质点为2个自由度。	
\end{example}
\begin{example}	
	质点$M$被限定只能在球面的上半部分,${{(x-a)}^{2}}+{{(y-b)}^{2}}+{{(z-c)}^{2}}={{R}^{2}}$,这样质点在空间中的位置就由两个独立参数所确定,即它的自由度数为2。
\end{example}